\documentclass[12pt]{article}
\usepackage[a4paper, portrait, margin=0.8in]{geometry}
\usepackage{graphicx}

\author{Aurélien Casteilla}
\date{26-04-2021}
\title{Spiderchip64}

\begin{document}
\pagenumbering{gobble}
\maketitle
\begin{figure}[h!]
    \includegraphics[width=\linewidth]{mascotte.png}
\end{figure}
\newpage

\paragraph{}
        Spiderchip64 is a 64 bits little endian ISA. This ISA tries to apply
        the KISS principle and also tries to respect the Popek and Goldberg
        virtualization requirement. Also, this CPU will include atomic memory
        operation for a multi-core version. So Spiderchip64 is a RISC ISA.
        Spiderchip64 has a 64 bits-wide data bus, a 64 bits-wide address bus,
        two sets of 31 64 bits-wide registers for a total of 47 registers, plus
        a zero register.  There is also a status register and a saved status
        register. One set of register is used by the user's program and the
        other set is reserved for fast interrupt handling or for the supervisor
        usage. The instruction word is 32 bit wide.
        \setcounter{secnumdepth}{-1}
\section{Summary}
\begin{itemize}
    \item RISC ISA
    \item applies the KISS principle
    \item respects the Popek and Goldberg virtualization requirement
    \item have atomic memory operations
    \item 64 bits or less address bus (more than you will ever need (16 EiB))
    \item 64 bits data bus
    \item 64 bits registers
    \item 32 bits instruction word
    \item 15 registers shared between the user and the supervisor
    \item 16 registers are user specific and 16 registers are supervisor specific 
    \item 32 registers visible at the same time (including a zero register)
    \item one address space for everything (Von Neumann architecture with I/O mapped in memory)
    \item "user" and "supervisor" mode
    \item one instruction per clock cycle if the architecture is pipelined    
	   (excluding memory instructions especially if the memory is slow)
    \item SIMD instructions on 2 words, 4 half-words or 8 bytes
    \item free and open source
\end{itemize}
\section{Notes :}
\begin{itemize}
    \item This ISA is intended to be implemented the first time on a FPGA
        development board (Arduino MKR Vidor 4000, with an Intel Cyclone 10)

    \item I am aware that Intel Quartus is not free software. I am searching a
        FOSS FPGA "compiler" and FOSS-friendly FPGA board

    \item This is just an ISA for the CPU part of a computer. So it doesn't
        have and doesn't specify a MMU, a FPU, a GPU, a cache, a PIC, or an I/O
        controller. However, some of these elements will be on the FPGA SoC.

    \item In the first version of this ISA, there won't be double instructions.

\end{itemize}
\section{Glossary of the acronyms :}

\paragraph{CPU : }Central Processing Unit
\paragraph{FPGA : }Field Programmable Gate Array
\paragraph{FPU : }Floating Point Unit
\paragraph{FOSS : }Free and Open Sourced Software
\paragraph{GPU : }Graphical Processing Unit
\paragraph{I/O : }Input / Output
\paragraph{ISA : }Instruction Set Architecture
\paragraph{KISS : }Keep It Simple Stupid
\paragraph{MIMD : }Multiple Instruction Multiple Data 
\paragraph{MMU : }Memory Management Unit
\paragraph{PIC : }Programmable Interrupt Controller
\paragraph{RISC : }Reduced Instruction Set Computing
\paragraph{SoC : }System On Chip

\section{}
Cyclone 10 is a trademark owned by Intel

\end{document}
