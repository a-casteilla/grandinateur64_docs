\documentclass[11pt]{article}
\usepackage[a4paper, portrait, margin=0.8in]{geometry}
%\usepackage[T1]{inputenc}
\usepackage{multirow}
\usepackage{graphicx}

\author{Aurélien Casteilla}
\title{Instruction set architecture of Spiderchip64 \\[1ex] \large Version 0.1}
\date{14-05-2021}

\begin{document}


\tolerance=1
\emergencystretch=\maxdimen
\hyphenpenalty=10000
\hbadness=10000

\pagenumbering{gobble}
\maketitle
\begin{figure}[h!]
    \includegraphics[width=\linewidth]{mascotte.png}
\end{figure}
\newpage

\tableofcontents
\newpage

\pagenumbering{arabic}
\section{Register file}
\subsection{Regular registers}

%   \paragraph{User register set :} \newline
\begin{table}[h!]
    \begin{center}
        \caption{User register set}
        \label{tab:urs}
        \begin{tabular}{|l|l|l|l|l|l|l|l|}
            \hline
            R0  & R1    & R2    & R3    & R4    & R5    & R6    & R7 \\
            \hline
            R8    & R9    & R10   & R11   & R12   & R13   & R14  & PC \\
            \hline
            R16  & R17    & R18    & R19    & R20    & R21    & R22    & R23 \\
            \hline
            R24    & R25    & R26   & R27   & R28   & R29   & R30   & LR \\
            \hline
        \end{tabular}
    \end{center}
\end{table}

%   \paragraph{Interrupt register set :} \newline
\begin{table}[h!]
    \begin{center}
        \caption{Interrupt register set}
        \label{tab:irs}
        \begin{tabular}{|l|l|l|l|l|l|l|l|}
            \hline
            R0  & R1    & R2    & R3    & R4    & R5    & R6    & R7 \\
            \hline
            R8    & R9    & R10   & R11   & R12   & R13   & R14  & PC \\
            \hline
            RLIi & R17i    & R18i    & R19i    & R20i    & R21i    & R22i    & R23i \\
            \hline
            R24i    & R25i    & R26i   & R27i   & R28i   & R29i   &  R30i  & LRi \\
            \hline
        \end{tabular}
    \end{center}
\end{table}

R0 is hard-wired to zero : loading a value from it returns 0,
storing a value in it does nothing. \newline
RLIi : Last instruction executed before the interrupt\newline
LR : link register\newline
LRi : supervisor link register\newline
PC : program counter, always point the next instruction to be executed\newline
Registers R0 to R14 and PC are shared between the sets. \newline
All registers are 64 bit-wide\newline

\newpage
\subsection{Status register and conditional code register}
The whole status register is directly accessible only in supervisor mode, with the instructions TRS and TSR.
In user mode, the current condition codes can be accessed with the instructions TRC and TCR.
The condition codes bits are bit 7 to bit 0. The condition codes bits can be used to perform conditional
operations, such as branches. 
All interrupts move the current status register to the saved status register. And RTI moves back the saved status
to the current status. RTI can be executed only in supervisor mode. 

\begin{table}[h!]
    \begin{center}
        \caption{Current status register and saved status register}
        \label{tab:status}
        \begin{tabular}{|c|c|c|l|}
            \hline
            & bit No. & name & description \\
            \hline
            unused & 63-32 & - & \\
            \hline
            \multirow{16}{*}{saved status} & 31 & R & interrupt register set selected \\
            & 30 &I& interrupt request disabled \\
            & 29 &D& map directly the memory\\
            & 28 &W& wait interrupt \\
            & 27 &P&privileged instructions allowed \\
            & 26 &-& \\
            & 25 &-& \\
            & 24 &-& \\
            & 23 &N&negative \\
            & 22 &V&overflow \\
            & 21 &-& \\
            & 20 &H&higher than (H = C./Z) \\
            & 19 &g&greater than or equal (g = /N./V + N.V) \\
            & 18 &G&greater than (G = /N./V./Z + N.V./Z) \\
            & 17 &Z&zero \\
            & 16 &C&carry \\
            \hline
            \multirow{8}{*}{current status}& 15 &R& interrupt register set selected \\
            & 14 &I& interrupt request disabled \\
            & 13 &D& map directly the memory\\
            & 12 &W& wait interrupt \\
            & 11 &P&privileged instructions allowed \\
            & 10 &-& \\
            & 9 &-& \\
            & 8 &-& \\
            \hline
            \multirow{8}{*}{current status and conditional codes} & 7 &N&negative \\
            & 6 &V&overflow \\
            & 5 &-& \\
            & 4 &H&higher than (H = C./Z) \\
            & 3 &g&greater than or equal (g = /N./V + N.V) \\
            & 2 &G&greater than (G = /N./V./Z + N.V./Z) \\
            & 1 &Z&zero \\
            & 0 &C&carry \\
            \hline
        \end{tabular}
    \end{center}
\end{table}
\begin{itemize}
    \item R : if set, R14i to LRi are used instead of R14 to LR
    \item I : if set, interrupt requests are disabled 
    \item D : if set, the MMU doesn't translate the addresses
    \item W : if set, the CPU waits until an interrupt is received. If I is
        set, the processor continues the execution without affecting the PC nor
        the status (except W which is cleared). Otherwise, it performs a
        regular interrupt.
    \item P : if set, the CPU can perform privileged instructions and memory access
    \item N : this status is set if the result of an instruction is negative.
    \item V : this status is set if a signed overflow occurs. A signed overflow has occurred
        when the result has not the expected sign.
    \item H : this status is set if the unsigned result is strictly higher than zero in case of
        a subtraction
    \item g : this status is set for a signed subtraction if the first operand is greater than
        or equal the second operand. 
    \item G : this status is set for a signed subtraction if the first operand is strictly 
        greater than the second operand. 
    \item Z : this status is set when the result is zero.
    \item C : this status is set when an addition has an unsigned result that overflows. In 
        case of a subtraction, this status is set if the unsigned result is higher than or equal 
        than zero. 
\end{itemize}

\newpage
\section{Interrupts}

\paragraph{For all interrupts : }
The CPU starts executing instruction at a predefined address, swaps the register set, 
and saves  the current PC in LRi, saves the current status to the saved status and 
dumps the last instruction tried into the RLIi register.
The status bits D, I, R and P are set, and W is cleared. An interrupt servicing routine has to
be quitted with the instruction RTI to restore the PC and the saved status.
For multicore systems, \$100 multiplied by the core number is added to the jump address.
The highest priority number is the least prioritised interrupt.

\begin{table}[h!]
    \begin{center}
        \caption{interrupt table}
        \label{tab:int}
        \begin{tabular}{|l|c|c|p{30mm}|p{70mm}|}
            \hline
            name & jump address & Priority & short description & long description \\ 
            \hline
            Reset & \texttt{\$FFFFFFFF FFFFFFFC} & 0 & Reset & This interrupt performs a cold reset of the system.
            This interrupt has to be triggered on power on. After the reset,
            only the core 0 is started. For the other cores, the status and the
            PC are initialised, but the W status is left set.\\
            \hline
            ME    & \texttt{\$00000000 00000000} & 2 & memory error &
            This interrupt is triggered when the MMU detects an error or a privilege violation.
            It aborts the current instruction.
            The faulty instruction is dumped into RLIi. \newline
            \textbf {Warning :} we assume that no memory error will occur during an interrupt servicing
            routine. However, if it happens anyway, the CPU will overwrite the LRi. (But if a memory error
            happen in an ISR, your system is probably already really f**ked up) \\
            \hline
            AE   & \texttt{\$00000000 00000004} & 3 & Address error & 
            This interrupt happen when the processor tries to access an address not aligned.
            It aborts the current instruction.
            The faulty instruction is dumped into RLIi. \newline
            \textbf{Warning :} we assume that no address error will occur during an interrupt servicing
            routine. However, if it happens anyway, the CPU will overwrite the LRi. 
            Thus, address error should be ABSOLUTELY avoided in an ISR. \\
            \hline
            ILG   & \texttt{\$00000000 00000008} & 4 & Illegal instruction &This interrupt happen when an undefined 
            instruction is executed or privileged instruction in user mode is executed
            The faulty instruction is dumped into RLIi. \\
            \hline
            SYS   & \texttt{\$00000000 0000000C} & 4 & System call & This interrupt happen when the SYS opcode is executed. 
            The SYS opcode is stored in the RLIi. \\
            \hline
            IRQ   & \texttt{\$00000000 00000010} & 1 & Interrupt request&This interrupt is the general 
            purpose hardware interrupt. This interrupt is maskable with the status I.
            The last instruction executed is dumped in the RLIi. \\
            \hline
        \end{tabular}
    \end{center}
\end{table}

\newpage
\section{Addressing modes}

\subsection{Memory operations}
In this section EA stands for effective address

\paragraph{direct mode : (d)}

offset (15 bits signed --- 32 KiB range)

EA = offset

\paragraph{index mode : (i)}

Rx, offset (10 bits signed --- 1 KiB range)

EA = Rx + offset

\paragraph{base index shifted mode : (bi) }

Rx, Ry \textless\textless shift (shift amount on 5 bits)

EA = Rx + Ry \textless\textless shift

\paragraph{auto-indexing mode : (ai) }

\subparagraph{Rx, Ry+}

Ry += data\_size

\subparagraph{Rx, Ry-}

EA = Rx + Ry then Ry -= data\_size

\subparagraph{Rx, +Ry}

Ry += data\_size then EA = Rx + Ry

\subparagraph{Rx, -Ry}

Ry -= data\_size then EA = Rx + Ry


\paragraph{Important notes : }
\begin{itemize}
    \item Every memory access has to be aligned. If not, a MAM interrupt is raised. 
        \begin{itemize}
            \item If a double-word is accessed, it must be aligned on double-word
                (The least significant nibble is 0 or 8). 
             \item if a word is accessed, it must be word aligned (the least significant nibble of
                the address has to be either 0, 4, 8 or C). 
            \item If it is a half-word, the address has to be even.
            \item For a byte, every address can be accessed. 
        \end{itemize}

    \item The PC is a valid register for addressing. In addition, the PC is always word 
        aligned. 

    \item The endianess is little. 
\end{itemize}

\subsection{ALU operations}

\paragraph{register mode : (rr)}
\texttt{Rx, Ry [<< <left\_shift\_amount (4 bits)>]}

S1 = Rx, S2 = Ry (left shifted by left\_shift\_amount)

\paragraph{immediate mode : (rim)}
\texttt{Rx, \#<immediate\_value>}

S1 = Rx, S2 = immediate\_value (9 bits signed)

\subsection{Branch operations}

\paragraph{relative to register mode : (r)}
offset

PC = Rx + offset (14 bits signed left shifted by 2 --- 64 KiB range)

\newpage
\section{Instruction set}
\subsection{Memory operations}

\paragraph{LDW} --- Load double-word \newline
Loads a double-word into a register from memory

\paragraph{LWS} --- Load signed word \newline
Loads a word into a register from memory
Extends the sign bit.

\paragraph{LWU} --- Load unsigned word\newline
Loads a word into a register from memory
Pads the most significant bits with 0.

\paragraph{LHS} --- Load signed half-word \newline
Loads a half-word into a register from memory
Extends the sign bit.

\paragraph{LHU} --- Load unsigned half-word\newline
Loads a half-word into a register from memory
Pads the most significant bits with 0.

\paragraph{LBS} --- Load signed byte\newline
Loads a byte into a register from memory
Extends the sign bit.

\paragraph{LBU} --- Load unsigned byte\newline
Loads a byte into a register from memory
Pads the most significant bits with 0.

\paragraph{STD} --- Store double-word\newline
Stores a double-word into the memory from a register 
(will cause a LLD / SCD on the same address to fail)

\paragraph{STW} --- Store word\newline
Stores a word into the memory from a register 
stores only the least 32 least significant bits
(will cause a LLD / SCD on the same address to fail)

\paragraph{STH} --- Store half-word\newline
Stores a half-word into the memory from a register
stores only the least 16 least significant bits
(will cause a LLD / SCD on the same address to fail)

\paragraph{STB} --- Store byte\newline
Stores a byte into the memory from a register
stores only the least 8 least significant bits
(will cause a LLD / SCD on the same address to fail)

\paragraph{Notes : }
For all memory operations, the suffix S can be added to update the status

\newpage
\subsection{Atomic memory operations}

\paragraph{LLD} --- load link double-word\newline
Loads a double-word from memory to a register. The load address is
stored in a hidden register, the load link register, and the dirty bit of this
register is cleared. If a store occures on this address, the dirty bit will be
set.

\paragraph{SCD} --- store conditional double-word\newline
Stores a double-word in memory from a register if the address of the target is
in the load link register, and if the dirty bit is cleared. If V is set after
the operation, the store has failed. Otherwise, it has succeeded. This
operation will cause another LLD / SCD on the same address to fail. If the
memory wasn't locked with LLD, the operation will fail. A core can't handle
more than one LLD / SCD at a time.

\paragraph{Notes : }
For all memory operations, the suffix S can be added to update the status

\subsection{Generic ALU operations}
General syntax for the documentation :
\newline
General 3 operands form : \texttt{FOO D, S1, S2}

\paragraph{ADD} --- Add\newline
Does \newline D = S1 + S2

\paragraph{SUB} --- Subtract\newline
Does \newline D = S1 - S2

\paragraph{AND} --- Bitwise and\newline
Does \newline D = S1 \& S2

\paragraph{ORR} --- Bitwise or\newline
Does \newline D = S1 | S2

\paragraph{EOR} --- Bitwise exclusive or\newline
Does \newline D = S1 \textasciicircum S2

\paragraph{ASL / LSL} --- Arithmetic shift left / Logical shift left S2 times\newline
Does \newline D = S1 \textless \textless  S2
\newline
\texttt{C<-xxxx...xxxx<-0}

\paragraph{ASR} --- Arithmetic shift right S2 times\newline
Does \newline D = S1 \textgreater \textgreater  S2\newline  where S1 is signed
\newline
\texttt{+->xxxx...xxxx->C}\newline
\texttt{|~~|             }\newline
\texttt{+--+             }

\paragraph{LSR} --- Logical shift right S2 times\newline
Does \newline D = S1 \textgreater \textgreater  S2\newline  where S1 is unsigned
\newline
\texttt{0->xxxx...xxxx->C}

\paragraph{ROL} --- rotate left S2 times\newline
Does \newline D = S1 ROL S2\newline
\texttt{+-xxxx...xxxx<-+->C}\newline
\texttt{|~~~~~~~~~~~~~~|}\newline
\texttt{+--------------+}

\paragraph{ROR} --- rotate right S2 times\newline
Does \newline D = S1 ROR S2\newline
\texttt{+->xxxx...xxxx-+->C}\newline
\texttt{|~~~~~~~~~~~~~~|}\newline
\texttt{+--------------+}

\paragraph{MUL} --- Multiply\newline
Does \newline D = S1 * S2 \newline
\textbf{Warning : } only the 32 least significant bits of the operands 
are multiplied together to give a 64 bits result. 

\paragraph{Notes : }
For all ALU operations, the suffix S can be added to update the status

\subsection{Carried ALU operations}

\paragraph{ADC} --- Add with carry\newline
Does \newline D = S1 + S2 + C\newline  where C is 1 if the carry is set, 0 otherwise

\paragraph{SBC} --- Subtract with borrow\newline
Does \newline D = S1 - S2 + C - 1\newline  where C is 1 if the carry is set, 0 otherwise

\paragraph{RLC} --- rotate left through carry S2 times\newline
Does \newline D = S1 RLC S2\newline
\texttt{C<-xxxx...xxxx<-C}

\paragraph{RRC} --- rotate right through carry S2 times\newline
Does \newline D = S1 RRC S2\newline
\texttt{C->xxxx...xxxx->C}

\paragraph{Notes : }
For all ALU operations, the suffix S can be added to update the status

\newpage
\subsection{SIMD ALU operations}

For all SIMD operations, if S2 is immediate, then each S2 word, half-word or byte
will be the same. E.g : \newline
This : \newline
\texttt{ADDH R3, R3, \#-3}\newline
Is the same as this : \newline
\texttt{ADDH R3, R3, R4 ; if R4 = \$FFFDFFFD FFFDFFFD}\newline

\subsubsection{SIMD add}

\paragraph{ADDD} --- Add double-words (is the same as ADD)\newline

\paragraph{ADDW} --- Add words\newline
Adds S1 and S2 to D, but doesn't carry the bit 31.

\paragraph{ADDH} --- Add half-words\newline
Adds S1 and S2 to D, but doesn't carry the bits 47, 31 and 15.

\paragraph{ADDB} --- Add bytes\newline
Adds S1 and S2 to D, but doesn't carry for the bits 55, 47, 39, 31, 23, 15 and
7.

\subsubsection{SIMD subtract}

\paragraph{SUBD} --- Subtract double-words (is the same as SUB)\newline

\paragraph{SUBW} --- Subtract words\newline
Subtracts S1 and S2 to D, but doesn't carry the bit 31.

\paragraph{SUBH} --- Subtract half-words\newline
Subtracts S1 and S2 to D, but doesn't carry the bits 47, 31 and 15.

\paragraph{SUBB} --- Subtract bytes\newline
Subtracts S1 and S2 to D, but doesn't carry for the bits 55, 47, 39, 31, 23, 15
and 7.

\subsubsection{SIMD arithmetical or logical shift left}

\paragraph{ASLD} --- Shift double-words (is the same as ASL)\newline

\paragraph{ASLW} --- Shift words\newline
Shifts S1 and S2 to D, but doesn't carry the bit 31.

\paragraph{ASLH} --- Shift half-words\newline
Shifts S1 and S2 to D, but doesn't carry the bits 47, 31 and 15.

\paragraph{ASLB} --- Shift bytes\newline
Shifts S1 and S2 to D, but doesn't carry for the bits 55, 47, 39, 31, 23, 15
and 7.

\subsubsection{SIMD arithmetical shift right}

\paragraph{ASRD} --- Shift double-words (is the same as ASR)\newline

\paragraph{ASRW} --- Shift words\newline
Shifts S1 and S2 to D, but doesn't carry to the bit 31.

\paragraph{ASRH} --- Shift half-words\newline
Shifts S1 and S2 to D, but doesn't carry the bits 47, 31 and 15.

\paragraph{ASRB} --- Shift bytes\newline
Shifts S1 and S2 to D, but doesn't carry for the bits 55, 47, 39, 31, 23, 15
and 7.

\subsubsection{SIMD logical shift right}

\paragraph{LSRD} --- Shift double-words (is the same as LSR)\newline

\paragraph{LSRW} --- Shift words\newline
Shifts S1 and S2 to D, but doesn't carry to the bit 31.

\paragraph{LSRH} --- Shift half-words\newline
Shifts S1 and S2 to D, but doesn't carry the bits 47, 31 and 15.

\paragraph{LSRB} --- Shift bytes\newline
Shifts S1 and S2 to D, but doesn't carry for the bits 55, 47, 39, 31, 23, 15
and 7.

\subsubsection{SIMD rotate left}

\paragraph{ROLD} --- Shift double-words (is the same as ROL)\newline

\paragraph{ROLW} --- Shift words\newline
Shifts S1 and S2 to D, but wrap around bit 63 to bit 32 and bit 31 to bit 0.

\paragraph{ROLH} --- Shift half-words\newline
Shifts S1 and S2 to D, but wrap around bit 63 to bit 48, bit 47 to bit 32, bit
31 to bit 16 and bit 15 to bit 0.

\paragraph{ROLB} --- Shift bytes\newline
Shifts S1 and S2 to D, but wrap around bit 63 to bit 56, bit 55 to bit 48, bit
47 to bit 40, bit 39 to bit 32, bit 31 to bit 24, bit 23 to bit 16, and bit 15
to bit 8, and bit 7 to bit 0.

\subsubsection{SIMD rotate right}

\paragraph{RORD} --- Shift double-words (is the same as ROR)\newline

\paragraph{RORW} --- Shift words\newline
Shifts S1 and S2 to D, but wrap around bit 32 to bit 63 and bit 0 to bit 31.

\paragraph{RORH} --- Shift half-words\newline
Shifts S1 and S2 to D, but wrap around bit 48 to bit 63, bit 32 to bit 47, bit
16 to bit 31 and bit 0 to bit 15.

\paragraph{RORB} --- Shift bytes\newline
Shifts S1 and S2 to D, but but wrap around bit 56 to bit 63, bit 48 to bit 55,
bit 40 to bit 47, bit 32 to bit 39, bit 24 to bit 31, bit 16 to bit 23, and bit
8 to bit 15, and bit 0 to bit 7.

\subsubsection{SIMD sum}

\paragraph{SUMW} --- And words\newline
Sums each word of S2 then adds it S1 to D.

\paragraph{SUMH} --- And half-words\newline
Sums each half-word of S2 then adds it S1 to D.

\paragraph{SUMB} --- And bytes\newline
Sums each byte of S2 then adds it S1 to D.

\paragraph{Notes : }
\begin{itemize}
    \item For all SIMD ALU operations, the suffix S can be added to update the status. 
        However, these are pointless because it works the same way for double-words.
    \item For the SIMD barrel shifting operations, each shift amount are independents. 
\end{itemize}

\subsection{Branch instructions}

\paragraph{BAL} --- branch always\newline
Add a displacement to the PC.

\paragraph{BEQ} --- branch on equal\newline
The branch is taken if Z is set

\paragraph{BNE} --- branch on not equal\newline
The branch is taken if Z is cleared

\paragraph{BMI} --- branch on minus\newline
The branch is taken if N is set

\paragraph{BPL} --- branch on plus\newline
The branch is taken if N is cleared

\paragraph{BVS} --- branch on overflow set\newline
The branch is taken if V is set

\paragraph{BVC} --- branch on overflow cleared\newline
The branch is taken if V is cleared

\paragraph{BCS / BHQ} --- branch on carry set / branch on higher than or equal
(Unsigned)\newline
The branch is taken if C is set

\paragraph{BCC / BLO} --- branch on carry cleared / branch on lower than
(Unsigned)\newline
The branch is taken if C is cleared

\paragraph{BHI} --- branch on higher than (Unsigned)\newline
The branch is taken if H is set

\paragraph{BLQ} --- branch on lower or equal (Unsigned)\newline
The branch is taken if H is cleared

\paragraph{BGT} --- branch on greater than (Signed)\newline
The branch is taken if G is set

\paragraph{BLE} --- branch on less than or equal (Signed)\newline
The branch is taken if G is cleared

\paragraph{BGE} --- branch on greater than or equal (Signed)\newline
The branch is taken if g is set

\paragraph{BLT} --- branch on less than (Signed)\newline
The branch is taken if g is cleared

\paragraph{BLK} --- branch and link\newline
This instruction branches then saves the old PC to LR. \newline
The purpose of this instruction is to branch to a subroutine. 

\subsection{Conditional codes instructions}

\paragraph{TCR} --- transfer conditional codes to register\newline
This instruction transfers the conditional codes byte to a register. \newline

\paragraph{TRC} --- transfer register to conditional codes\newline
This instruction transfers a register to the conditional codes byte. \newline

\subsection{Status instructions (privileged)}

\paragraph{TSR} --- transfer status to register\newline
This instruction transfers the whole status word to a register. \newline
This instruction will raise an ILG interrupt if it is executed in user mode.

\paragraph{TRS} --- transfer register to status\newline
This instruction transfers a register to the whole status word. \newline
This instruction will raise an ILG interrupt if it is executed in user mode.

\paragraph{RTI} --- return from interrupt\newline
The purpose of this instruction is to end an interrupt servicing routine and 
return to the application program. It copies the LRi register into the PC and
transfers back the saved status to the current status. \newline
This instruction will raise an ILG interrupt if it is executed in user mode.

\subsection{Software interrupts}

\paragraph{SYS} --- system call (software interrupt)\newline
Interrupts the processor to execute the system call interrupt handler

\paragraph{ILG} --- Illegal instruction\newline
Interrupts the processor to execute the illegal instruction interrupt handler

\subsection{conditional prefix}
A conditional prefix can be added to all instructions.

\paragraph{AL.} --- Always (equivalent to no prefix) \newline
Always executed.

\paragraph{EQ.} --- Equal\newline
Executed if Z is set. 

\paragraph{NE.} --- Non-equal\newline
Executed if Z is cleared. 

\paragraph{MI.} --- Minus\newline
Executed if N is set. 

\paragraph{PL.} --- Plus\newline
Executed if N is cleared. 

\paragraph{VS.} --- Overflow set\newline
Executed if V is set. 

\paragraph{VC.} --- Overflow cleared\newline
Executed if V is cleared. 

\paragraph{CS. / HQ.} --- carry set / higher or equal (unsigned)\newline
Executed if C is set. 

\paragraph{CC. / LO.} --- carry cleared / lower than (unsigned)\newline
Executed if C is cleared. 

\paragraph{HI.} --- higher than (unsigned)\newline
Executed if H is set. 

\paragraph{LQ.} --- lower than or equal (unsigned)\newline
Executed if H is cleared. 

\paragraph{GT.} --- greater than\newline
Executed if G is set. 

\paragraph{LE.} --- less than or equal\newline
Executed if G is cleared. 

\paragraph{GE.} --- greater than or equal\newline
Executed if g is set. 

\paragraph{LT.} --- less than\newline
Executed if g is cleared. 

\subsection{Meta-instructions}

\paragraph{MOV} --- move data\newline
Alias of \texttt{ADD <Rd> , R0, <Rx|\#immediate\_value> }

\paragraph{MVN} --- move negative\newline
Alias of \texttt{SUB <Rd> , R0, <Rx|\#immediate\_value> }

\paragraph{NOT} --- move 1st complement\newline
Alias of \texttt{EOR <Rd> , <Rx> , \#-1}

\paragraph{CMP} --- compare\newline
Alias of \texttt{SUBS R0, <Rx> , <Ry|\#immediate\_value> }

\paragraph{CMN} --- compare negative\newline
Alias of \texttt{ADDS R0, <Rx> , <Ry|\#immediate\_value> }

\paragraph{BIT} --- test bits\newline
Alias of \texttt{ANDS R0, <Rx> , <Ry|\#immediate\_value> }

\paragraph{NOP} --- the traditional No OPeration\newline
Alias of \texttt{ADD R0, R0, R0}

\paragraph{PSH} --- push onto the stack\newline
Alias of \texttt{STD <Rd>, -R30}

\paragraph{PLL} --- pull from the stack\newline
Alias of \texttt{LDW <Rd>, R30+}

\paragraph{RTS} --- return from subroutine\newline
Alias of \texttt{MOV PC, LR}

\paragraph{Note : }

Like ALU operations, the suffix S can be added to update the status

\newpage
\section{Assembler syntax}
\subsection{Memory operations}
Rd is the register source in case of a store, Rd is the destination register
in case of a load. The next values are for the addressing mode.
options between \textless \textgreater mean required, options between [] mean optional
\begin{itemize}
    \item \texttt{d : FOO <Rd>, <offset(15 bits)> }
    \item \texttt{i : FOO <Rd>, <Rx> [, offset(11 bits)] }
    \item \texttt{bi: FOO <Rd>, <Rx>, <Ry> [, shift(6 bits)]}
    \item \texttt{ai: FOO <Rd>, <Rx>+}
    \item \texttt{ai: FOO <Rd>, <Rx>-}
    \item \texttt{ai: FOO <Rd>, +<Rx>}
    \item \texttt{ai: FOO <Rd>, -<Rx>}
\end{itemize}
\subsection{ALU operations}
Rd is the destination register. The next values are for the addressing mode.
Options between \textless \textgreater mean required, options between [] mean
optional

\begin{itemize}
    \item \texttt{rr : FOO <Rd>, <Rx>, <Ry> [<LSL|ASL|ROL|LSR|ASR|ROR>
        <shift\_amount>]}
    \item \texttt{rim: FOO <Rd>, <Rx>, \#<immediate\_value(12
        bits)>}
\end{itemize}
General syntax for the documentation :
\newline
General form : \texttt{FOO D, S1, S2}

\subsection{Branch operations}
The values are for the addressing mode.
Options between \textless \textgreater mean required, options between [] mean
optional
For 3 operands : 
\begin{itemize}
    \item \texttt{i : BAR <Rx> [, offset]}
    \item \texttt{i : BAR [PC,] <offset>}
\end{itemize}
\subsection{Status instructions and conditional codes instructions}

Rd is the destination for TSR and TCR, and Rd is the source for TRS and TCR.
RTI does not have any addressing mode.
\begin{itemize}
    \item \texttt{implied : TRC <Rd>}
    \item \texttt{implied : TCR <Rd>}
    \item \texttt{implied : TRS <Rd>}
    \item \texttt{implied : TSR <Rd>}
    \item \texttt{none : RTI }
\end{itemize}

\subsection{Software interrupts}

Software interrupts don't have any addressing mode. 
However, a 19 bits numeric constant can be added to a SYS opcode. 
\begin{itemize}
    \item \texttt{none : SYS [\#constant]}
    \item \texttt{none : ILG }
\end{itemize}

\end{document}
