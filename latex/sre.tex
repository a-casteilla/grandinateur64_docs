\documentclass[11pt]{article}
\usepackage[a4paper, portrait, margin=0.8in]{geometry}
%\usepackage[T1]{inputenc}
\usepackage{multirow}
\usepackage{graphicx}

\author{Aurélien Casteilla}
\title{Simple runtime environnement \\[1ex] \large Version 0.1}
\date{14-05-2021}

\begin{document}

\maketitle
\newpage

\section{Overview}
The Simple runtime environnement (or SRE) is a machine with a bare Mansion64
CPU, 640 Kibibytes of RAM, 320 Kibibytes of ROM and an ASCII or ANSI terminal.
The user and the supervisor have almost the same rights except that only the
supervisor can access the terminal. The ROM is in the high memory and the RAM
is in the low memory. This computer is only intended for CPU testing purpose.

\section{Memory map}

The address bus is 20 bits wide, so there is only 1 mebiobyte of address space.

\begin{table}[h!]
    \begin{center}
        \caption{Address map}
        \label{tab:addmap}
        \begin{tabular}{|r|c|l|}
            \hline
            Address range & Memory type & Description \\
            \hline
            \texttt{\$00000 - \$00014} & RAM & Interrupt vectors \\
            \hline
            \texttt{\$00010 - \$08000} & RAM & Memory area available for customised
            interrupt handler \\
            \hline
            \texttt{\$08000 - \$a0000} & RAM & Free space \\
            \hline
            \texttt{\$a0000 - \$a0008} & Output & Terminal output (character
            at the byte \texttt{\$a0000})\\
            \hline
            \texttt{\$a0008 - \$a0010} & Input & Terminal input (character at the byte
            \texttt{\$a0008}) \\ 
            \hline
            \texttt{\$a0010 - \$b0000} & Unpopulated & Crash area \\
            \hline
            \texttt{\$b0000 - \$f8000} & ROM & Program and initial data \\
            \hline
            \texttt{\$f8000 - \$fffff} & ROM & Memory area available for the reset
            routine and \\
            & & for the default interrupt handlers \\
            \hline
            \texttt{\$ffffc - \$fffff} & ROM & Reset vector \\
            \hline
        \end{tabular}
    \end{center}
\end{table}

\section{Terminal}

A character is written on the terminal when a byte is written to the address
\texttt{\$A0000}. \newline
A character is read from the terminal when a byte is read from the
address \texttt{\$A0008}. \newline
When a character is available an IRQ happen. The IRQ is released when the
character is read from the terminal.

\section{Interrupts}
\subsection{Reset}
This computer does a reset only on power up.
\subsection{Memory error}
Any write operation on the ROM leads a memory error. Any access between address
\texttt{\$A0010} and \texttt{\$AFFFF} will do also a memory error. If the user tries to access
addresses between \texttt{\$A0000} and \texttt{\$A000F}, a memory error will occure as well.
\subsection{Address error}
An address error happen when a non-aligned address is accessed.
\subsection{IRQ}
An IRQ happen when a character arrived from the terminal. The IRQ is cleared
when a read operation happen on the terminal input address.

\end{document}
